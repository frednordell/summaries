\documentclass[12pt]{article} % Default font size is 12pt, it can be changed here

\usepackage[utf8]{inputenc} % utf8 encoding
\usepackage{geometry} % Required to change the page size to A4
\geometry{a4paper} % Set the page size to be A4 as opposed to the default US Letter

\usepackage{graphicx} % Required for including pictures
\usepackage{float} % Allows putting an [H] in \begin{figure} to specify the exact location of the figure

\linespread{1.2} % Line spacing

%\setlength\parindent{0pt} % Uncomment to remove all indentation from paragraphs

\graphicspath{{pictures/}} % Specifies the directory where pictures are stored

\usepackage{listings} % To be able to have code in text
\usepackage{color} % To be able to have colors

\definecolor{codegreen}{rgb}{0,0.6,0}
\definecolor{codegray}{rgb}{0.5,0.5,0.5}
\definecolor{codepurple}{rgb}{0.58,0,0.82}
\definecolor{backcolour}{rgb}{0.95,0.95,0.92}
 
\lstdefinestyle{codestyle}{
    backgroundcolor=\color{backcolour},   
    commentstyle=\color{codegreen},
    keywordstyle=\color{magenta},
    numberstyle=\tiny\color{codegray},
    stringstyle=\color{codepurple},
    basicstyle=\footnotesize,
    breakatwhitespace=false,         
    breaklines=true,                 
    captionpos=b,                    
    keepspaces=true,                 
    numbers=left,                    
    numbersep=5pt,                  
    showspaces=false,                
    showstringspaces=false,
    showtabs=false,                  
    tabsize=2
}
 
\lstset{style=codestyle}

\begin{document}

%----------------------------------------------------------------------------------------
%   TITLE PAGE
%----------------------------------------------------------------------------------------

\begin{titlepage}

\newcommand{\HRule}{\rule{\linewidth}{0.5mm}} % Defines a new command for the horizontal lines, change thickness here

\center % Center everything on the page

\textsc{\LARGE Lund University, Faculty of Engineering}\\[1.5cm] % Name of your university/college
\textsc{\Large EDAA40}\\[0.5cm] % Major heading such as course name
\textsc{\large Discrete Strucures in Computer Science}\\[0.5cm] % Minor heading such as course title

\HRule \\[1cm]
{ \huge \bfseries Summary of EDAA40}\\[0.4cm] % Title of your document
\HRule \\[1.5cm]

\emph{Author:} Fred \textsc{Nordell} % Your name

{\large \today}\\[3cm] % Date, change the \today to a set date if you want to be precise

%\includegraphics{Logo}\\[1cm] % Include a department/university logo - this will require the graphicx package

\vfill % Fill the rest of the page with whitespace

\end{titlepage}

%----------------------------------------------------------------------------------------
%   TABLE OF CONTENTS
%----------------------------------------------------------------------------------------

\tableofcontents % Include a table of contents
\lstlistoflistings % Include a table of lstlistings
\listoffigures % Include listing of figures
\listoftables

\newpage % Begins the essay on a new page instead of on the same page as the table of contents 


\section{Sets} % Major section

Sets are collections of stuff, any stuff. There is one special set, the empty set: $\{\} = \emptyset$. Given a set A, any ting x is either an element of A or it is not; $x \in A$ or $x \notin A$. To make this work we need a concept of equality, is $\{2,1\} \in \{\{1,2\},\{3,4\}\}$ ? 

\subsection{Extensionallity}
A set is defined by the elements it contains (its \textit{extension}). Order and repetition does not matter. $\{a,b,c\} = \{c,b,a\} = \{a,a,b,c,b\}$ Equal sets must contain \textit{exactly} the same elements $\{a,b,c\} \neq \{a,c,b,d\}$. Note that 1-element sets are \textit{singleton} sets: $\{a,b,c\} \neq \{\{a,b,c\}\}$ \& $\emptyset \neq \{\emptyset\}$ \& $11 \neq \{11\}$.

\subsection{Cardinallity}

The number of elements in a set $A$ is called its cardinallity. $\#(A)$ or $|A|$. Note that $\#(\emptyset) = 0$

\subsection{Inclusion}
If $A \subseteq B$ we call A a \textit{subset} of B and B the \textit{superset} of A. This means that if $x \in A$ then $x \in B$. A and B might be the same, in fact: $A \subseteq B$ and $B \subseteq A$ iff (if and only if) $A = B$. Furthermore, for any set A it is allways the case that $\emptyset \subseteq A$ and $A \subseteq A$. Note that $\subset$ is used to denote \textit{proper} inclusion: $A \subset B$ iff $A \subseteq B$ and $A \neq B$.

\subsubsection{Properties of inclusion}
Inclusion is \textit{transitive}: $A \subseteq B$ and $B \subseteq C$ implies $A \subseteq C$. Inclusion is also \textit{partial}: There are sets where neither $A \subseteq B$ or $B \subseteq A$ is true.

\subsection{Specifying sets}
There are a couple of ways to define sets: enumeration, set builder and recursive defenition.

\subsubsection{Enumeration}
Simple, just list the elements: $A = \{2,3,4,5,7,11\}$

\subsubsection{Builder notaion}
Little more complicated: $B = \{x \in A: x \quad odd\} \rightarrow B = \{3,5,7,11\}$. This uses other sets to select different elements based on a condidtion. 

\par Here we find \textit{Russel's paradox}, that is $R = \{x: x\notin x\}$ but does $R \in R$? Here $R$ isn't a well-defined set, $x$ can stand for anything, even things that aren't a set. Because of this we make sure that varibles are limited to elements of a set that is known to be well-defined. This form automatically implies a superset.

\subsubsection{Recursive}

\subsection{Operations on sets}
\subsubsection{Union}
The union, $A \cup B$,of $A$ and $B$ is all elements that are in $A$ or $B$. That is: $x \in A \cup B$ iff $x \in A$ or $x \in B$

\subsubsection{intersection}
The intersection, $A \cap B$, of $A$ and $B$ is all elements that are both in $A$ and $B$. That is: $x \in A \cap B$ iff $x \in A$ and $x \in B$.

\subsubsection{difference}
The difference, $A \setminus B$, of $A$ and $B$ is all elements that are in $A$ and not in $B$. That is: $x \in A \setminus B$ iff $x \in A$ and $x \notin B$.

\begin{figure}[H] % Example image
\center{\includegraphics[width=0.5\linewidth]{operation.jpg}}
\caption{Illustration of set operations}
\label{operations}
\end{figure}

\subsection{Complement}
There is no general \"inverse\" set $-A$ for the set $A$. However we often work in a \textit{local universe}, that is a set of everything we are potentionally interested in. Let's call it $U$. We can con give the complement of a set a meaning: $-A = U \setminus A$. Can also be written $A^- \quad A' \quad A^c$



\end{document}