\documentclass[12pt]{article} % Default font size is 12pt, it can be changed here

\usepackage[utf8]{inputenc} % utf8 encoding
\usepackage{geometry} % Required to change the page size to A4
\geometry{a4paper} % Set the page size to be A4 as opposed to the default US Letter

\usepackage{amssymb}
\def\R{\mathbb{R}}
\def\N{\mathbb{N}}
\usepackage{amsmath}
\usepackage{graphicx} % Required for including pictures
\usepackage{float} % Allows putting an [H] in \begin{figure} to specify the exact location of the figure

\linespread{1.2} % Line spacing

%\setlength\parindent{0pt} % Uncomment to remove all indentation from paragraphs

\graphicspath{{pictures/}} % Specifies the directory where pictures are stored

\usepackage{listings} % To be able to have code in text
\usepackage{color} % To be able to have colors

\definecolor{codegreen}{rgb}{0,0.6,0}
\definecolor{codegray}{rgb}{0.5,0.5,0.5}
\definecolor{codepurple}{rgb}{0.58,0,0.82}
\definecolor{backcolour}{rgb}{0.95,0.95,0.92}
 
\lstdefinestyle{codestyle}{
    backgroundcolor=\color{backcolour},   
    commentstyle=\color{codegreen},
    keywordstyle=\color{magenta},
    numberstyle=\tiny\color{codegray},
    stringstyle=\color{codepurple},
    basicstyle=\footnotesize,
    breakatwhitespace=false,         
    breaklines=true,                 
    captionpos=b,                    
    keepspaces=true,                 
    numbers=left,                    
    numbersep=5pt,                  
    showspaces=false,                
    showstringspaces=false,
    showtabs=false,                  
    tabsize=2
}
 
\lstset{style=codestyle}

\begin{document}

%----------------------------------------------------------------------------------------
%   TITLE PAGE
%----------------------------------------------------------------------------------------

\begin{titlepage}

\newcommand{\HRule}{\rule{\linewidth}{0.5mm}} % Defines a new command for the horizontal lines, change thickness here

\center % Center everything on the page

\textsc{\LARGE Lund University, Faculty of Engineering}\\[1.5cm] % Name of your university/college
\textsc{\Large EDAA40}\\[0.5cm] % Major heading such as course name
\textsc{\large Discrete Strucures in Computer Science}\\[0.5cm] % Minor heading such as course title

\HRule \\[1cm]
{ \huge \bfseries Summary of EDAA40}\\[0.4cm] % Title of your document
\HRule \\[1.5cm]

\emph{Author:} Fred \textsc{Nordell} % Your name

{\large \today}\\[3cm] % Date, change the \today to a set date if you want to be precise

%\includegraphics{Logo}\\[1cm] % Include a department/university logo - this will require the graphicx package

\vfill % Fill the rest of the page with whitespace

\end{titlepage}

%----------------------------------------------------------------------------------------
%   TABLE OF CONTENTS
%----------------------------------------------------------------------------------------

\tableofcontents % Include a table of contents
\lstlistoflistings % Include a table of lstlistings
\listoffigures % Include listing of figures
\listoftables

\newpage % Begins the essay on a new page instead of on the same page as the table of contents 


\section{Sets} % Major section

Sets are collections of stuff, any stuff. There is one special set, the empty set: $\{\} = \emptyset$. Given a set A, any ting x is either an element of A or it is not; $x \in A$ or $x \notin A$. To make this work we need a concept of equality, is $\{2,1\} \in \{\{1,2\},\{3,4\}\}$ ? 

\subsection{Extensionallity}
A set is defined by the elements it contains (its \textit{extension}). Order and repetition does not matter. $\{a,b,c\} = \{c,b,a\} = \{a,a,b,c,b\}$ Equal sets must contain \textit{exactly} the same elements $\{a,b,c\} \neq \{a,c,b,d\}$. Note that 1-element sets are \textit{singleton} sets: $\{a,b,c\} \neq \{\{a,b,c\}\}$ \& $\emptyset \neq \{\emptyset\}$ \& $11 \neq \{11\}$.

\subsection{Cardinallity}

The number of elements in a set $A$ is called its cardinallity. $\#(A)$ or $|A|$. Note that $\#(\emptyset) = 0$

\subsection{Inclusion}
If $A \subseteq B$ we call A a \textit{subset} of B and B the \textit{superset} of A. This means that if $x \in A$ then $x \in B$. A and B might be the same, in fact: $A \subseteq B$ and $B \subseteq A$ iff (if and only if) $A = B$. Furthermore, for any set A it is allways the case that $\emptyset \subseteq A$ and $A \subseteq A$. Note that $\subset$ is used to denote \textit{proper} inclusion: $A \subset B$ iff $A \subseteq B$ and $A \neq B$.

\subsubsection{Properties of inclusion}
Inclusion is \textit{transitive}: $A \subseteq B$ and $B \subseteq C$ implies $A \subseteq C$. Inclusion is also \textit{partial}: There are sets where neither $A \subseteq B$ or $B \subseteq A$ is true.

\subsection{Specifying sets}
There are a couple of ways to define sets: enumeration, set builder and recursive defenition.

\subsubsection{Enumeration}
Simple, just list the elements: $A = \{2,3,4,5,7,11\}$

\subsubsection{Builder notaion}
Little more complicated: $B = \{x \in A: x \quad odd\} \rightarrow B = \{3,5,7,11\}$. This uses other sets to select different elements based on a condidtion. 

\par Here we find \textit{Russel's paradox}, that is $R = \{x: x\notin x\}$ but does $R \in R$? Here $R$ isn't a well-defined set, $x$ can stand for anything, even things that aren't a set. Because of this we make sure that varibles are limited to elements of a set that is known to be well-defined. This form automatically implies a superset.

\subsubsection{Recursive}

\subsection{Operations on sets}
\subsubsection{Union}
The union, $A \cup B$,of $A$ and $B$ is all elements that are in $A$ or $B$. That is: $x \in A \cup B$ iff $x \in A$ or $x \in B$

\subsubsection{intersection}
The intersection, $A \cap B$, of $A$ and $B$ is all elements that are both in $A$ and $B$. That is: $x \in A \cap B$ iff $x \in A$ and $x \in B$.

\subsubsection{difference}
The difference, $A \setminus B$, of $A$ and $B$ is all elements that are in $A$ and not in $B$. That is: $x \in A \setminus B$ iff $x \in A$ and $x \notin B$.

\begin{figure}[H]
\center{\includegraphics[width=0.5\linewidth]{operation.jpg}}
\caption{Illustration of set operations}
\label{operations}
\end{figure}

\subsubsection{Complement}
There is no general \"inverse\" set $-A$ for the set $A$. However we often work in a \textit{local universe}, that is a set of everything we are potentionally interested in. Let's call it $U$. We can con give the complement of a set a meaning: $-A = U \setminus A$. Can also be written $A^- \quad A' \quad A^c$

\subsection{Disjointness}
Two sets are \textit{disjoint} if they do not share any common elements: $A \cap B = \emptyset$. All sets are disjoint from $\emptyset$, event the empty set. For multiple sets they are \textit{pairwise disjoint} if they are all disjoint.

\subsection{Algebra}

\begin{figure}[H]
\center{\includegraphics[width=0.5\linewidth]{algebra.png}}
\caption{Illustration of set algebra}
\label{set algebra}
\end{figure}

\subsection{Set families}

A family of sets is a way of reffering to a set of sets, usually \textit{indexed} by an \textit{index set}. $\{A_i : i \in I\}$, where $A_i$ are sets, $i$ is the index and $I$ is the index set.

\subsubsection{Generalized union and intersection}
Let $S$ be a set of sets then $\bigcap S = \{x : x \in s$ for all $s \in S\}$ and $\bigcup S = \{ x : x \in s$ for at least one $s \in S\}$

\subsection{Power sets}

The \textit{power set} of a set A is the set of all its subsets. $\mathcal{P}(A) = \{s : s \subseteq A \} = 2^A$

\begin{itemize}
    \item $\mathcal{P}(\emptyset) = \emptyset$
    \item $\mathcal{P}(\{a\}) = \{\emptyset, \{a\}\} $
    \item $\mathcal{P}(\{a,b\}) = \{\emptyset, \{a\}, \{b\}, \{a,b\}\}$
    \item $\mathcal{P}(\{a,b,c\}) = \{\emptyset, \{a\},\{b\},\{c\},\{a,b\},\{a,c\},\{b,c\},\{a,b,c\}\}$
    \item $\#(\mathcal{P}(A)) = 2^{\#(A)}$ 
\end{itemize}

\begin{figure}[H]
\center{\includegraphics[width=0.5\linewidth]{hasse.png}}
\caption{Hasse diagram of powerset}
\label{hasse diagram}
\end{figure}

\section{Relations}

Mathematical relation are about connections between objects. It can be relations between numbers: a divides b, a is greater than etc. or it can be relations between sets: A is a subset of B, same size etc. or it can be between people: customer/client, parent/child.

\subsection{Ordered pairs}
An ordered pair is diffeent from an unordered pair wich we have seen before in that the flipped pair is not the same. $(a,b) = (x,y)$ iff $a = x$ and $b = y$, corollary: $(a,b) \neq (b,a)$ if $a \neq b$. N-tuples is this idea but with n number of elements in the tulple.

\subsection{Cartesian product}
The cartesian product of a pair of sets, or generally a finite family of sets, is the set of all ordered pais or n-tuples.
\[
    A_1 \times ... \times A_n = \{(a_1 , ...,a_n) : a_1 \in A_1, ..., a_n \in A_n\}
\]
If sets are the same we also write $A \times A = A^2$

\begin{figure}[H]
\center{\includegraphics[width=0.5\linewidth]{cartesian.png}}
\caption{Examples of cartesian product}
\label{Cartesian product examples}
\end{figure}

\subsection{Relation and natural join}

A (binary, dyadic) relation R from A to B, or over $A \times B$, is a subset of the cartesian product.

\[
    R \subseteq A \times B
\]

If A and B are the same i.e. $R \subseteq A \times A$ we call R a binary relation over A. For binary relations $R \subseteq A \times B$, these are equvalent: $(a,b) \in R$ and $aRb$. The $\bowtie$ operator is the \textit{natural join}, that gives the set of all combinations of tuples in the relations, $R \bowtie S$, taht are equal on their commn attribute names. It is important because it is the relational counterpart to the logical AND.

\begin{figure}[H]
\center{\includegraphics[width=0.5\linewidth]{naturaljoin.png}}
\caption{Examples of natural join}
\label{Natural join example}
\end{figure}

\subsection{Source, target, domain, range}
For binary relations $R \subseteq A \times B$: A is called the \textit{source} and B is called the \textit{target}. Note that source and target are not uniquely determined as:
\[
    Y \supseteq A \quad X \supseteq B
\]

\[
    A \times B \subseteq X \times Y 
\]

\[
    R \subseteq A \times B \subseteq X \times Y
\]

By contrast the \textit{domain}, $dom(R) = \{a : (a,b) \in R \ \textrm{for some } \ b\}$, and the \textit{range}, $range(R) = \{b : (a,b) \in R \ \textrm{for some} \ a\}$


\subsection{Converse and complement}
For a binary relation $R \subseteq A \times B$ its \textit{converse (inverse)} is the relation: $R^{-1} = \{(b,a) : aRb\}$. That is, the relation but with $a$ and $b$ swapped.

\begin{figure}[H]
\center{\includegraphics[width=0.5\linewidth]{converse.png}}
\caption{Some properties of the converse of a relation}
\label{Converse properties}
\end{figure}

FOr a binary relation $R \subseteq A \times B$ its \textit{complement} is the relation $\bar{R} = -_{A \times B}R = A \times B \setminus R$. That is, everyting in the cartesian product except the relation itself.

\begin{figure}[H]
\center{\includegraphics[width=0.5\linewidth]{complement.png}}
\caption{Some properties of the complement of a relation}
\label{Conplement properties}
\end{figure}

\begin{figure}[H]
\center{\includegraphics[width=0.5\linewidth]{cvc.png}}
\caption{Converse vs complement}
\label{Converse vs complement}
\end{figure}

\subsection{Composistion}
Given two binary relations $R \subset A \times B$ and $S \subset B \times C$ their \textit{composition} is a binary relation on $A \times C$, or more precise: $S \circ R = \{(a,c) : aRb$ and $bSc$ for some $b \in B\}$ 


\begin{figure}[H]
\center{\includegraphics[width=0.5\linewidth]{composition.png}}
\caption{Composition visualised}
\label{Composition visualised}
\end{figure}

\subsection{Image}

Given a binary relation $R \subseteq A \times B$ from A to B, for any $a \in A$ its \textit{image under r}, written $R(a)$, is defined as: $R(a) = \{b \in B : aRb\}$. 

\subsection{Reflexivity, transitivity, symmetry}

\section{Functions}
A function is a special case of a relation, the relation $f \subseteq A \times B$ is a function iff $dom(f) = A$ and $\#(f(a)) = 1$ for all $a \in A$. This is written $A \rightarrow B$

\begin{figure}[H]
\center{\includegraphics[width=0.5\linewidth]{rvf.png}}
\caption{Relations vs Functions}
\label{Relations vs Functions}
\end{figure}

The set of all functions from A to B can be written as $\langle A \rightarrow B \rangle$ or $B^A$. So these are the same: $f: A \rightarrow B$, $f \in \rangle A \rightarrow B \langle$, $f \in B^A$

\subsection{Mapping}
In addition to domain and codomain, we need to describe the actual mapping defining the function. We use this arrow $\mapsto$ for this. $f: A \rightarrow B$ and $x \mapsto$ (something with $x$), here the first part denotes the domain and codomain and the second part describes what we actually do with the values in the domain.

\subsubsection{functions of multiple arguments}
Functions of multiple arguments are simply functions of Cartesian products:
\[
    add: \R \times \R \rightarrow \R
\]
\[
    (x,y) \mapsto x + y
\]

But usually to simplify the parenthesis are removed 
\[
    add: \R \times \R \rightarrow \R
\]
\[
    x,y \mapsto x + y
\]

This also applies to using a function, we write $add(5,7)$ instead of $add((5,7))$

\subsection{Restriction}
Given a function $f: A \to B$ its \textit{restriction} to a set $X \subseteq A$ is defined as $fx: X \to B$, $a \mapsto f(a)$, alternatively wirtten as $f | x$

\subsection{Image}
Given a function $f: A \to B$ and a set $X \subseteq A$, the image of $X$ under f is defined as: $f(X) = \{f(a): a \in X\}$

\subsection{Injection, surjection, bijection}

\paragraph{Injection} A function $f: A \to B$ is injective iff: $a \neq b$ implies $f(a) \neq f(b)$. Notation: $f: A \hookrightarrow B$.

\paragraph{Surjective} A function $f: A \to B$ is surjective iff $f(A) = B$. Notation: $f: A \twoheadrightarrow B$

\paragraph{Bijective} A function $f: A \to B$ is bijective iff it is both injective and surjective. Notation: $f: A \leftrightarrow B$

\section{Proofs}
A \textit{definition} is a statement that gives a precise meaning ti a term or a symbol. A \textit{theorem} is astatement that needs to be proven based on definitions (and axioms). A \textit{proof} is a chain of logical reasoning showing the truth of a theorem.

\subsection{Kinds of proof}
Proofs come in different forms, which one is used depends on the problem and what chain of reasoning is best suited. Many theorems are conditional statements, that is they have the form \"premise implies conclusion\" or $P \Rightarrow C$ e.g. if $x$ is odd, then $x^2$ is odd.

\begin{table}[h!]
\centering
    \begin{tabular}{ | c | c | c |}
      \hline
      P & C &  $P \Rightarrow C$ \\
      \hline
      T & T & T \\
      T & F & F \\
      F & T & T \\
      F & F & T \\
      \hline
    \end{tabular}
\caption{Truthtable}
\label{table: 1}
\end{table}

\subsubsection{Direct proof}

\subsubsection{Contrapositive}

In some cases it is easier to prove a theorem by supposing that the conclusion is false and then prove that the premise then also is false $\neg C \Rightarrow \neq P$

\begin{table}[h!]
\centering
    \begin{tabular}{ | c | c | c |}
      \hline
      $\neg P$ & $\neg C$ &  $P \Rightarrow C$ \\
      \hline
      F & F & T \\
      T & F & F \\
      F & T & T \\
      T & T & T \\
      \hline
    \end{tabular}
\caption{Contrapositive truthtable}
\label{table: 2}
\end{table}

\subsubsection{Proof by contradiction}

Say we want to prove a proposition, not necessarialy in conditional form. Proof by contradiction uses the fact that if we cn show that $\neg P$ results in a logical contradiction, e.g. it implies some conclusion $C$ as well ass $\neg C$ then $\neg P$ cannot be true, hence $P$ must be true.

\begin{table}[h!]
\centering
    \begin{tabular}{ | c | c | c | c | c |}
      \hline
      $P$ & $C$ &  $\neg P$ & $C \land \neg C$ & $\neg P \Rightarrow C \land \neg C$\\
      \hline
      T & T & F & F & T \\
      T & F & F & F & T\\
      F & T & T & F & F\\
      F & F & T & F & F\\
      \hline
    \end{tabular}
\caption{Contradictory truthtable}
\label{table: 3}
\end{table}

\section{Induction and recursion}

\subsection{Induction} Induction is the method of proving the properties of things.

\begin{figure}[H]
\center{\includegraphics[width=0.5\linewidth]{induction.png}}
\caption{An example of an induction proof}
\label{Induction example}
\end{figure}

\begin{figure}[H]
\center{\includegraphics[width=0.5\linewidth]{generalised_induction.png}}
\caption{Generalised induction proof}
\label{Induction generalised}
\end{figure}

\subsection{Recursion} Recursion is used to define sets, and espessially functions. A simple recursive defenition follows.

\[
f(n) =
    \begin{cases}
        1 & \text{for } n = 0\\
        n \cdot f(n - 1) & \text{otherwise}\\ 
    \end{cases}
\]
This describes the following set: 

\[
    f= \{(0,1),(1,1),(2,2),(3,6),(4,24),(5,120),\ldots\}
\]

A culumative recursive definition reach back further than the last defined value, possibly to all defined values. THis can also be extended to the induction principle as we now assumed the truth for \textbf{all} smaller values, not just the pervious one.

\section{Propositional logic}

\begin{figure}[H]
\center{\includegraphics[width=0.5\linewidth]{logic.png}}
\caption{Basic logic functions}
\label{Basic logic functions}
\end{figure}



\end{document}