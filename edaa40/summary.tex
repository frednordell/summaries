\documentclass[12pt]{article} % Default font size is 12pt, it can be changed here

\usepackage[utf8]{inputenc} % utf8 encoding
\usepackage{geometry} % Required to change the page size to A4
\geometry{a4paper} % Set the page size to be A4 as opposed to the default US Letter

\usepackage{graphicx} % Required for including pictures
\usepackage{float} % Allows putting an [H] in \begin{figure} to specify the exact location of the figure

\linespread{1.2} % Line spacing

%\setlength\parindent{0pt} % Uncomment to remove all indentation from paragraphs

\graphicspath{{pictures/}} % Specifies the directory where pictures are stored

\usepackage{listings} % To be able to have code in text
\usepackage{color} % To be able to have colors

\definecolor{codegreen}{rgb}{0,0.6,0}
\definecolor{codegray}{rgb}{0.5,0.5,0.5}
\definecolor{codepurple}{rgb}{0.58,0,0.82}
\definecolor{backcolour}{rgb}{0.95,0.95,0.92}
 
\lstdefinestyle{codestyle}{
    backgroundcolor=\color{backcolour},   
    commentstyle=\color{codegreen},
    keywordstyle=\color{magenta},
    numberstyle=\tiny\color{codegray},
    stringstyle=\color{codepurple},
    basicstyle=\footnotesize,
    breakatwhitespace=false,         
    breaklines=true,                 
    captionpos=b,                    
    keepspaces=true,                 
    numbers=left,                    
    numbersep=5pt,                  
    showspaces=false,                
    showstringspaces=false,
    showtabs=false,                  
    tabsize=2
}
 
\lstset{style=codestyle}

\begin{document}

%----------------------------------------------------------------------------------------
%   TITLE PAGE
%----------------------------------------------------------------------------------------

\begin{titlepage}

\newcommand{\HRule}{\rule{\linewidth}{0.5mm}} % Defines a new command for the horizontal lines, change thickness here

\center % Center everything on the page

\textsc{\LARGE Lund University, Faculty of Engineering}\\[1.5cm] % Name of your university/college
\textsc{\Large EDAA40}\\[0.5cm] % Major heading such as course name
\textsc{\large Discrete Strucures in Computer Science}\\[0.5cm] % Minor heading such as course title

\HRule \\[1cm]
{ \huge \bfseries Summary of EDAA40}\\[0.4cm] % Title of your document
\HRule \\[1.5cm]

\emph{Author:} Fred \textsc{Nordell} % Your name

{\large \today}\\[3cm] % Date, change the \today to a set date if you want to be precise

%\includegraphics{Logo}\\[1cm] % Include a department/university logo - this will require the graphicx package

\vfill % Fill the rest of the page with whitespace

\end{titlepage}

%----------------------------------------------------------------------------------------
%   TABLE OF CONTENTS
%----------------------------------------------------------------------------------------

\tableofcontents % Include a table of contents

\newpage % Begins the essay on a new page instead of on the same page as the table of contents 


\section{Sets} % Major section

Sets are collections of stuff, any stuff. There is one special set, the empty set: $\{\} = \emptyset$. Given a set A, any ting x is either an element of A or it is not; $x \in A$ or $x \notin A$. To make this work we need a concept of equality, is $\{2,1\} \in \{\{1,2\},\{3,4\}\}$ ? 

\subsection{Extensionallity}
A set is defined by the elements it contains (its \textit{extension}). Order and repetition does not matter. $\{a,b,c\} = \{c,b,a\} = \{a,a,b,c,b\}$ Equal sets must contain \textit{exactly} the same elements $\{a,b,c\} \neq \{a,c,b,d\}$. Note that 1-element sets are \textit{singleton} sets: $\{a,b,c\} \neq \{\{a,b,c\}\}$ \& $\emptyset \neq \{\emptyset\}$ \& $11 \neq \{11\}$.

\subsection{Cardinallity}

The number of elements in a set $A$ is called its cardinallity. $\#(A)$ or $|A|$. Note that $\#(\emptyset) = 0$

\subsection{Inclusion}
If $A \subseteq B$ we call A a \textit{subset} of B and B the \textit{superset} of A. This means that if $x \in A$ then $x \in B$. A and B might be the same, in fact: $A \subseteq B$ and $B \subseteq A$ iff (if and only if) $A = B$. Furthermore, for any set A it is allways the case that $\emptyset \subseteq A$ and $A \subseteq A$. Note that $\subset$ is used to denote \textit{proper} inclusion: $A \subset B$ iff $A \subseteq B$ and $A \neq B$.

\subsubsection{Properties of inclusion}
Inclusion is \textit{transitive}: $A \subseteq B$ and $B \subseteq C$ implies $A \subseteq C$. Inclusion is also \textit{partial}: There are sets where neither $A \subseteq B$ or $B \subseteq A$ is true.

\subsection{Specifying sets}
There are a couple of ways to define sets,

\section{Content Section} % Major section

Some more content

\subsection{Subsection 1} % Sub-section

Some more content

\subsubsection{Subsubsection 2} % Sub-sub-section

Some more content

\begin{lstlisting}[language=Python]
import numpy as np
 
def incmatrix(genl1,genl2):
    m = len(genl1)
    n = len(genl2)
    M = None #to become the incidence matrix
    VT = np.zeros((n*m,1), int)  #dummy variable
 
    #compute the bitwise xor matrix
    M1 = bitxormatrix(genl1)
    M2 = np.triu(bitxormatrix(genl2),1) 
 
    for i in range(m-1):
        for j in range(i+1, m):
            [r,c] = np.where(M2 == M1[i,j])
            for k in range(len(r)):
                VT[(i)*n + r[k]] = 1;
                VT[(i)*n + c[k]] = 1;
                VT[(j)*n + r[k]] = 1;
                VT[(j)*n + c[k]] = 1;
 
                if M is None:
                    M = np.copy(VT)
                else:
                    M = np.concatenate((M, VT), 1)
 
                VT = np.zeros((n*m,1), int)
 
    return M
\end{lstlisting}


\begin{thebibliography}{99} % Bibliography - this is intentionally simple in this template

\bibitem[Figueredo and Wolf, 2009]{Figueredo:2009dg}
Figueredo, A.~J. and Wolf, P. S.~A. (2009).
\newblock Assortative pairing and life history strategy - a cross-cultural
  study.
\newblock {\em Human Nature}, 20:317--330.
 
\end{thebibliography}

%----------------------------------------------------------------------------------------

\end{document}