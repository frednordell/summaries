\documentclass[12pt]{article} % Default font size is 12pt, it can be changed here

\usepackage[utf8]{inputenc} % utf8 encoding
\usepackage{geometry} % Required to change the page size to A4
\geometry{a4paper} % Set the page size to be A4 as opposed to the default US Letter

\usepackage{graphicx} % Required for including pictures
\usepackage{amsmath}
\usepackage{float} % Allows putting an [H] in \begin{figure} to specify the exact location of the figure

\linespread{1.2} % Line spacing

%\setlength\parindent{0pt} % Uncomment to remove all indentation from paragraphs

\graphicspath{{pictures/}} % Specifies the directory where pictures are stored

\usepackage{listings} % To be able to have code in text
\usepackage{color} % To be able to have colors

\definecolor{codegreen}{rgb}{0,0.6,0}
\definecolor{codegray}{rgb}{0.5,0.5,0.5}
\definecolor{codepurple}{rgb}{0.58,0,0.82}
\definecolor{backcolour}{rgb}{0.95,0.95,0.92}
 
\lstdefinestyle{codestyle}{
    backgroundcolor=\color{backcolour},   
    commentstyle=\color{codegreen},
    keywordstyle=\color{magenta},
    numberstyle=\tiny\color{codegray},
    stringstyle=\color{codepurple},
    basicstyle=\footnotesize,
    breakatwhitespace=false,         
    breaklines=true,                 
    captionpos=b,                    
    keepspaces=true,                 
    numbers=left,                    
    numbersep=5pt,                  
    showspaces=false,                
    showstringspaces=false,
    showtabs=false,                  
    tabsize=2
}
 
\lstset{style=codestyle}

\begin{document}

%----------------------------------------------------------------------------------------
%   TITLE PAGE
%----------------------------------------------------------------------------------------

\begin{titlepage}

\newcommand{\HRule}{\rule{\linewidth}{0.5mm}} % Defines a new command for the horizontal lines, change thickness here

\center % Center everything on the page

\textsc{\LARGE Lund University, Faculty of Engineering}\\[1.5cm] % Name of your university/college
\textsc{\Large FRTF05}\\[0.5cm] % Major heading such as course name
\textsc{\large Automatic Control, Basic Course}\\[0.5cm] % Minor heading such as course title

\HRule \\[1cm]
{ \huge \bfseries Summary of FRTF05}\\[0.4cm] % Title of your document
\HRule \\[1.5cm]

\emph{Author:} Fred \textsc{Nordell} % Your name

{\large \today}\\[3cm] % Date, change the \today to a set date if you want to be precise

%\includegraphics{Logo}\\[1cm] % Include a department/university logo - this will require the graphicx package

\vfill % Fill the rest of the page with whitespace

\end{titlepage}

%----------------------------------------------------------------------------------------
%   TABLE OF CONTENTS
%----------------------------------------------------------------------------------------

\tableofcontents % Include a table of contents
\lstlistoflistings % Include a table of lstlistings
\listoffigures % Include listing of figures
\listoftables % Include listing of tables

\newpage % Begins the essay on a new page instead of on the same page as the table of contents 


\section{Introduction} % Major section

Man använder reglerteknik för att styra saker, t.ex. temeraturen i et rum. Det är ett sätt att moddelera hur saker ska göra, typ. Det är fett användbart, så lära sig är bra.

\section{Regulering}

\subsection{Den enkla reglerkretsen}

Vi har en proces vi vill reglera: det vi mäter kallar vi $y$, mätsignal kallas också utsignal och ärvärde. Om vi vill kunna påverka har vi en pil in till $y$ som vi kallar $u$, styrsignal kallas också insignal. Innan $u$ sitter det vi kallar för regeluator, det är den som bestämmer, vad behöver den veta? 1. $r$ referensvärde eller börvärde, e.g. vilken temperatur vill vi ha. 2. $y$ mätsignalen, e.g. hur går det? 

\subsection{On/off-regulatorn}
Reglerfelet: $e = r -y$. En nackdel med on-off är 

\[
u =
    \begin{cases}
        u_{max} & \text{om } e \geq 0 \\
        u_{min} & \text{om } e \leq 0 \\ 
    \end{cases}
\]

\subsection{P-regulatorn}
Ekvation: $P: u = u_0 + K \cdot e$
\[
u =
    \begin{cases}
        u_{max} & \text{om } e \geq 0 \\
        u_0 + K \cdot e & -e_0 \leq e \leq e_0 \\
        u_{min} & \text{om } e \leq 0 \\ 
    \end{cases}
\]

K är regulatorns förstärkning. Problem, man får nästan alltid ett kvarstående reglerfel.
Vad är reglerfelet? $u = u_0 + K \cdot e$, $u(t) = u_0 + K \cdot e(t)$ detta get felet: $e = \dfrac{u - u_{0}}{K}$
Felet blir mindre om förstärkningen är större, men man svänger runt sitt önskade värde.

\subsection{PI-regulatorn}
$P: u = u_0 + K \cdot e$ men vad händer om vi reglerar $u$?
\[
    PI: u=\dfrac{K}{T_i}\int edt + k \cdot e = K(e + \dfrac{1}{T_i} \int edu)
\]

\subsection{PID-regulator}
PI-regulatorn har en begränsning: man måste kolla historiskt: ''vart är vi påväg?''

\[
    u(t) = K(e(t) + \dfrac{1}{T_i} \int e(t) dt + T_d \cdot \dfrac{de}{dt})
\] 
$T_d =$ regulatorns derrivatatid


\section{Föreläsning 2}

JAG HATAR MITT LIV, VARFÖR FINNS ALKOHOL AAAAAHHHHHH

Tillstånds beskrivning: 

\[
    \begin{cases}
        x = Ax + Bu \\
        y = Cx + Du \\
    \end{cases}
\]

nåt annat: $Y(s) = G(1)U(S)$

\subsection{Blockschemaräkning}

\[
    r \rightarrow \sum \rightarrow e \rightarrow G_R \rightarrow u \rightarrow G_p \rightarrow y
\]

y återkopplas med $-1$ till $\sum$ \\

$e = r - y$

\[
    b \rightarrow G \rightarrow a
\]


$A = GB$
laplace-transformer som fan

\[
    Y = G_p U
\]

\[
    U = G_r E
\]

\[
    E = R \cdot Y
\]

\[
    \rightarrow U = G_R (R -Y) \rightarrow Y = G_p G_r (R-Y) \rightarrow Y = \frac{G_p G_k}{1 + G_P G_k}R
\]

\subsection{svåra saker som jag inte vet}

\subsubsection{övergång tillståndsbeskrivning till G(s)}

Laplaca tillståndsbeskrivning, ropa på hjälp, gråt lite, hoppa av utbildningen.

\[
    \begin{cases}
        x = Ax + Bu \\
        y = Cx + Du \\
    \end{cases}
\]

\[
    \begin{cases}
        SX = AX + BU & \rightarrow SX - AX = BU \rightarrow (SI - A)X \text{Då s är skalär och A matris} \rightarrow X = (SI - A)^{-1} BU\\
        Y = CX + DU &\\
    \end{cases}
\]

\[
    Y = [C(SI - A^{-1}B + D)]U
\]


\subsubsection{Övergång G(s) till tillståndbeskrivning}

$$Y(s) = G(s)U(s)$$

Ej entydligt då det finns oändligt många lösningar. Ha lika många tillstånd som ordningen på övertranssaken.

Kolla i formelbladet nånting med kanonska formen sida 4: tillsvidare, återkommande, falafel.


\subsection{Impulssvar}

$$u(t) = \delta (t)$$

Överkurs:
$$U(s) = \int^\infty_0  e^{-st} u(t) dt = \int^\infty_0 e^{-st} \delta (t) dt = e^{-s0} = 1$$

$$Y(s) = G(s) \cdot U(s) = G(s)$$


Nu blir det valnigt igen: Stekpade? fan säger grabben? ah, stegsvar!

\[
u(t) = 
    \begin{cases}
        1 & t \geq 0\\
        0 & t < 0\\
    \end{cases}
\]

\[
    U(s) = \int^\infty_0 e^{-st} u(t) dt = \int^\infty_0 e^{-st} dt = \frac{1}{s} 
\]

$$Y(s) = G(s)U(s) =G(s) \cdot \frac{1}{s}$$


\textbf{Stegsvaret är integralen av impulssvaret}

\subsection{Samband överföringsfunktion och stegsvar.}

$$Y(s) = G(s)U(s) = G(s) \cdot \frac{1}{s}$$

begynnelsevärdeteoremet:

$$\lim_{t \to 0} f(t) = \lim_{s \to \infty} s F(s)$$

Slutvärdesteorem
$$\lim_{t \to \infty} f(t) = \lim_{s \to 0} s F(s)$$


\subsection{Ex:}

$$\lim_{t \to \infty} y(t) = \lim_{s \to 0} sY(s) = \lim_{s\to0} s \cdot G(s) \cdot \frac{1}{s} = \lim_{s\to0} G(s) = G(0)$$

$G(0)$ är den \textit{statiska förstärkningen}


\section{föreläsning 3, förvirringen fortsätter}

\subsection{frekvenssvar}

$$u(t) = sin(\omega t)$$

$$y(t) = a sin(\omega t + \phi)$$

$$a = |G(i\omega)|$$
$$\phi = arg G(i \omega)$$

Upprepa för flera frekvenser, svårt att se med bara en frekvens.
man får en tabell

\begin{tabular}{c | c |c }
    $\omega$ & a & $\phi$ \\
    \hline
    0.1 & 0.9 & $-10^{\circ}$ \\
    0.2 & 0.8 & $-20^{\circ}$ \\
    0.4 & 0.5 & $-60^{\circ}$\\
    0.8 & 0.2 & $-90^{\circ}$ \\
\end{tabular}


Gör man denna tabell kan man lite bakvägs få fram överföringsfunktionen, exprimentellt.

\subsection{Nyquistdiagram}

papper

\subsection{Bodediagram}

papper

%----------------------------------------------------------------------------------------

\end{document}