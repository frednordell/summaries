\documentclass[12pt]{article} % Default font size is 12pt, it can be changed here

\usepackage[utf8]{inputenc} % utf8 encoding
\usepackage{geometry} % Required to change the page size to A4
\geometry{a4paper} % Set the page size to be A4 as opposed to the default US Letter

\usepackage{graphicx} % Required for including pictures
\usepackage{amssymb}
\def\R{\mathbb{R}}
\def\N{\mathbb{N}}
\usepackage{amsmath}
\usepackage{float} % Allows putting an [H] in \begin{figure} to specify the exact location of the figure

\linespread{1.2} % Line spacing

%\setlength\parindent{0pt} % Uncomment to remove all indentation from paragraphs

\graphicspath{{pictures/}} % Specifies the directory where pictures are stored

\usepackage{listings} % To be able to have code in text
\usepackage{color} % To be able to have colors

\definecolor{codegreen}{rgb}{0,0.6,0}
\definecolor{codegray}{rgb}{0.5,0.5,0.5}
\definecolor{codepurple}{rgb}{0.58,0,0.82}
\definecolor{backcolour}{rgb}{0.95,0.95,0.92}
 
\lstdefinestyle{codestyle}{
    backgroundcolor=\color{backcolour},   
    commentstyle=\color{codegreen},
    keywordstyle=\color{magenta},
    numberstyle=\tiny\color{codegray},
    stringstyle=\color{codepurple},
    basicstyle=\footnotesize,
    breakatwhitespace=false,         
    breaklines=true,                 
    captionpos=b,                    
    keepspaces=true,                 
    numbers=left,                    
    numbersep=5pt,                  
    showspaces=false,                
    showstringspaces=false,
    showtabs=false,                  
    tabsize=2
}
 
\lstset{style=codestyle}

\begin{document}

%----------------------------------------------------------------------------------------
%   TITLE PAGE
%----------------------------------------------------------------------------------------

\begin{titlepage}

\newcommand{\HRule}{\rule{\linewidth}{0.5mm}} % Defines a new command for the horizontal lines, change thickness here

\center % Center everything on the page

\textsc{\LARGE Lund University, Faculty of Engineering}\\[1.5cm] % Name of your university/college
\textsc{\Large FMSF45}\\[0.5cm] % Major heading such as course name
\textsc{\large Matematisk statistik, allmän kurs}\\[0.5cm] % Minor heading such as course title

\HRule \\[1cm]
{ \huge \bfseries Summary of FMSF45}\\[0.4cm] % Title of your document
\HRule \\[1.5cm]

\emph{Author:} Fred \textsc{Nordell} % Your name

{\large \today}\\[3cm] % Date, change the \today to a set date if you want to be precise

%\includegraphics{Logo}\\[1cm] % Include a department/university logo - this will require the graphicx package

\vfill % Fill the rest of the page with whitespace

\end{titlepage}

%----------------------------------------------------------------------------------------
%   TABLE OF CONTENTS
%----------------------------------------------------------------------------------------

\tableofcontents % Include a table of contents
\lstlistoflistings % Include a table of lstlistings
\listoffigures % Include listing of figures
\listoftables % Include listing of tables

\newpage % Begins the essay on a new page instead of on the same page as the table of contents 


\section{Grundläggande berepp} % Major section

\paragraph{Utfall}
Resultatet av ett slumpmässigt försök. Betecknas: $\omega$
\paragraph{Händelse}
Samling av ett eller flera utfall. Betecknas: $A, B \ldots$
\paragraph{Utfallsrum}
Mängden av möjliga utfall. Betecknas: $\Omega$

\begin{itemize}
    \item Utfallsrum: $\Omega = \{1,2,3,4,5,6\}$
    \item utfall: $\omega_1 = 1, \omega_2 \ldots$
    \item Händelse: $A = \text{''Minst 4:a''} = \{\omega_4 , \omega_5 , \omega_6\} = \{4,5,6\}$
    \item 2 Tärningar (olika): Utfallsrum: $\Omega = \{(1,1),(1,2),(2,1), \ldots\}$ 36 fall
    \item 2 Tärningar (lika): Utfallsrum: $\Omega = \{(1,1),(1,2),(1,3), \ldots\}$ $\frac{6(6 + 1)}{2} = 21$ fall
    \item 2 Tärningar (samma): Utfallsrum: $\Omega = \{2,3, \ldots , 12\}$ 11 fall
\end{itemize}

\subsection{Venn-diagram}
Börja med en fyrkant, det är utfallsrummet. Ett utfall är en punkt i det rummet. En händelse är en samlign av utfall, en yta i diagrammet $A$. Vad är komplemetet till händelsen? dvs. icke $A$: betecknas $A^*$

\paragraph{Två händelser} 
Vad är: \\
$A$ eller $B$, det är $A \cup B$. \\
$A$ och $B$ är snittet $A \cap B$ \\
Ej överlappande $A$ och $B$, de är \textit{disjunkta}. $A \cap B = \emptyset$ \\

\subsection{Sannolikhet}
Betecknas $P(A)$.

\paragraph{Kolmogorovs axiomsystem}
\begin{itemize}
    \item $0 \leq P(A) \leq 1$
    \item $P(\Omega) = 1$
    \item $P(A \cup B) = P(A) + P(B)$
\end{itemize}

Komplementsatsen: $P(A^*) = 1 - P(A)$
\[
    P(A \cup A^*) = \{\text{disjunkta}\} = P(A) + P(A^*) = P(\Omega) = 1
\]

Additionssatsen:
$P(A \cup B) = P(A) + P(B) - P(A \cap B)$

\[
    A \cup B = A \cup (A^* \cap B)
\]

\[
    P(A \cup B) = P(A) + P(A^* \cap B)
\]

\[
    B = (A^* \cap B) \cup (A \cap B)
\]

\[
    P(B) = P(A^* \cap B) + P(A \cap B) \to P(A \cup B) = P(A) + P(B) - P(A \cap B)
\]

\subsection{Klassiska sannolikhetsdefinitionen}
Gynnsamma fall, g. möjliga fall, m:

\[
    P(A) = \frac{g}{m} \quad (= \frac{||A||}{||\Omega||}) \ ||A|| = \text{storleken på } A
\]

\[
    \binom{n}{k} = \frac{n!}{k! \cdot (n-k)!}
\]

\subsection{Betingad sannolikhet}
Om vi observerar $A$ hur sannolikt är $B$?

\[
    P(B | A) = \frac{P(A \cap B)}{P(A)}
\]

\[
    P(A \cap B) = P(B | A) \cdot P(A)
\]

\subsection{Satsen om total sannolikhet}
Om vi har $n$ händelser $H_1, \ldots, H_n$. \\
parvis disjunkta. \\
Täcker utfallsrummet: 
\[
    \bigcup_{i = 1}^{n} H_i = \Omega
\]

Gäller för varje händelse A att:

\[
    P(A) = \sum_{i = 1}^{n} P(A | H_i)P(H_i)
\]

\paragraph{Bayes sats}
\[
    P(H_i | A) = \frac{P(h_i \cap A)}{P(A)}
\]

\section{Mer förvirring}

Pallar inte ta anteckningar kl 8 på morgongen, se föreläsning 3








%----------------------------------------------------------------------------------------

\end{document}